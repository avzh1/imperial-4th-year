\documentclass[11pt]{article}

% some definitions for the title page
\newcommand{\reporttitle}{Deep learning for GPU poor}
\newcommand{\reportdescription}{}

% load some definitions and default packages
%---------------------------------------------------------------------------
%	PACKAGES AND OTHER DOCUMENT CONFIGURATIONS
%---------------------------------------------------------------------------

\usepackage[twoside]{fancyhdr}
\usepackage{csquotes}

\usepackage[a4paper,hmargin=2.0cm,vmargin=1.0cm,includeheadfoot]{geometry}
% \usepackage{natbib} % for bibliography
\usepackage{biblatex}
\usepackage{tabularx,longtable,multirow,subfigure,caption}%hangcaption
\usepackage{fancyhdr} % page layout
\usepackage{url} % URLs
\usepackage[english]{babel}
\usepackage{amsmath}
\usepackage{graphicx}
\usepackage{dsfont}
\usepackage{epstopdf} % automatically replace .eps with .pdf in graphics
% \usepackage{backref} % needed for citations
\usepackage{array}
\usepackage{latexsym}
\usepackage[pdftex,hypertexnames=false,colorlinks]{hyperref} % provide links in pdf (had pagebackref)
\usepackage{booktabs}
\usepackage{wrapfig}
\usepackage{caption}  % Required for \captionof
\usepackage{float} % for H option in figures
\usepackage{amssymb}
\usepackage{amsmath}
\usepackage[nottoc]{tocbibind}

%%% Default fonts
\renewcommand*{\rmdefault}{bch}
\renewcommand*{\ttdefault}{cmtt}

%%% Default settings (page layout)
\setlength{\parindent}{0em}  % indentation of paragraph
\setlength{\parskip}{.3em}

\setlength{\headheight}{14.5pt}
\pagestyle{fancy}

\fancyfoot[ER,OL]{\thepage}%Page no. in the left on odd pages and on right on even pages

\fancyfoot[OC,EC]{\sffamily }
\renewcommand{\headrulewidth}{0.1pt}
\renewcommand{\footrulewidth}{0.1pt}
\captionsetup{margin=10pt,font=small,labelfont=bf}
% Here, you can define your own macros. Some examples are given below.

\newcommand{\R}[0]{\mathds{R}} % real numbers
\newcommand{\Z}[0]{\mathds{Z}} % integers
\newcommand{\N}[0]{\mathds{N}} % natural numbers
\newcommand{\C}[0]{\mathds{C}} % complex numbers
\renewcommand{\vec}[1]{{\boldsymbol{{#1}}}} % vector
\newcommand{\mat}[1]{{\boldsymbol{{#1}}}} % matrix

\usepackage{pifont,mdframed}
\newenvironment{warning}
  {\par\begin{mdframed}[linewidth=1pt,linecolor=black]%
    \begin{list}{}{\leftmargin=1cm
                   \labelwidth=\leftmargin}\item[\Large\ding{43}]}
  {\end{list}\end{mdframed}\par}

%\bibliography{bibliography}

\begin{document}

% Include the title page
\begin{titlepage}

    \newcommand{\HRule}{\rule{\linewidth}{0.5mm}} % Defines a new command for the horizontal lines, change thickness here
    
    \center % Center everything on the page
     
    %------------------------------------------------------------------------
    %	HEADING SECTIONS
    %------------------------------------------------------------------------
    
    \textsc{\Large Department of Computing}\\[0.5cm] 
    \textsc{\large Imperial College of Science, Technology and Medicine}\\[0.5cm] 
    
    %------------------------------------------------------------------------
    %	TITLE SECTION
    %------------------------------------------------------------------------
    
    \HRule \\[0.4cm]
    { \huge \bfseries \reporttitle}\\ % Title of your document
    \HRule \\[0.4cm]

    \textit{\reportdescription}
    
    \vspace{2em}

    %------------------------------------------------------------------------
    %	AUTHOR SECTION
    %------------------------------------------------------------------------
    
    \large \emph{Author: Anton Zhitomirsky}

    \vspace{1em}

    \global\let\newpagegood\newpage
    \global\let\newpage\relax
    
\end{titlepage}

\global\let\newpage\newpagegood

\tableofcontents

\clearpage

\section{Calculating statistics of a model}

\subsection{Foundation model parameter count}

Commonly when you look online for models, you will see their parameter counts. \emph{Transformers are typically described by the \#parameters} which impacts the computational requirements to store and run these models.

\subsection{Floating Point System}

\begin{figure}[H]
    \centering
    \subfigure{\fbox{\includegraphics[page=4, trim=0cm 0cm 0cm 0cm, clip, width=.45\linewidth]{L15_deeplearning_for_gpu_poor_part_1.pdf}}}
    \subfigure{\fbox{\includegraphics[page=5, trim=0cm 0cm 0cm 0cm, clip, width=.45\linewidth]{L15_deeplearning_for_gpu_poor_part_1.pdf}}}
    \caption{We store data in computers in bits with transistors. The more bits, the more precision.}
\end{figure}    



\begin{minipage}[l]{.5\linewidth}
    \begin{figure}[H]
        \centering
        \subfigure{\fbox{\includegraphics[page=6, trim=0cm 0cm 0cm 0cm, clip, width=.95\linewidth]{L15_deeplearning_for_gpu_poor_part_1.pdf}}}
    \end{figure}    
\end{minipage}\hfill
\begin{minipage}[r]{.48\linewidth}
    \begin{itemize}
        \item \textbf{Floating point} is a way to represent numbers in a computer where the decimal point can float.
        \item A certain section is allocated to represent the precision (actual number) the exponent and the sign.
        \item still has the problem of underflow or overflow.
    \end{itemize}
\end{minipage}

\begin{minipage}[l]{.5\linewidth}
    \begin{figure}[H]
        \centering
        \subfigure{\fbox{\includegraphics[page=7, trim=0cm 0cm 0cm 0cm, clip, width=.95\linewidth]{L15_deeplearning_for_gpu_poor_part_1.pdf}}}
    \end{figure}    
\end{minipage}\hfill
\begin{minipage}[r]{.48\linewidth}
    \begin{itemize}
        \item At inference time, just to store the data it is already 260GB
        \item At training time, we also require gradients and optimiser values (momentum, etc) which can be 10x the size of the model and the activations.
    \end{itemize}
\end{minipage}

\begin{minipage}[l]{.5\linewidth}
    \begin{figure}[H]
        \centering
        \subfigure{\fbox{\includegraphics[page=8, trim=0cm 0cm 0cm 0cm, clip, width=.95\linewidth]{L15_deeplearning_for_gpu_poor_part_1.pdf}}}
    \end{figure}    
\end{minipage}\hfill
\begin{minipage}[r]{.48\linewidth}
    So if we pass through the network, we need to store the inputs to every single layer that theres a change in the computational graph. Here:
    \begin{itemize}
        \item The first is the batch times the number of tokens or the number of patches times by that dimensionality.
        \item For the query key value, they all get the same value.
        \item The attention matrix requires query with key-values
        \item We also need to store the $n^2$ complexity of the attention matrix. 
    \end{itemize}
\end{minipage}

\begin{itemize}
    \item Also we store the result of the soft max which is the batch times the number of heads times the sequence dimension.
    \item With multi-head attention we usually have another layer that learns how to combine these values together.
    \item All in all, we have 250GB for a batch size of 1. 
\end{itemize}

\begin{figure}[H]
    \centering
    \subfigure{\fbox{\includegraphics[page=9, trim=0cm 0cm 0cm 0cm, clip, width=.45\linewidth]{L15_deeplearning_for_gpu_poor_part_1.pdf}}}
\end{figure}

\subsection{Floating Point Operations (FLOPs)}

\begin{minipage}[l]{.5\linewidth}
    \begin{figure}[H]
        \centering
        \subfigure{\fbox{\includegraphics[page=10, trim=0cm 0cm 0cm 0cm, clip, width=.95\linewidth]{L15_deeplearning_for_gpu_poor_part_1.pdf}}}
    \end{figure}    
\end{minipage}\hfill
\begin{minipage}[r]{.48\linewidth}
    Computational requirements:
    \begin{itemize}
        \item given two vectors, if we add this will reuqire $n$ operations.
        \item For the dot products, there is $2n$ flops.
        \item For the matrix multiplication, there is $2nmp$ multiplications
    \end{itemize}
\end{minipage}

\begin{figure}[H]
    \centering
    \subfigure{\fbox{\includegraphics[page=11, trim=0cm 0cm 0cm 0cm, clip, width=.95\linewidth]{L15_deeplearning_for_gpu_poor_part_1.pdf}}}
\end{figure}    

\begin{enumerate}
    \item the layer norm is roughly 20 operations for every feature and token
    \item for the query key values we have it 3 times, and 2 becuase its a dot product.
    \item The softmax: for every single query we dot product it for every single value.
    \item projectino is same as query key values (but only 1)
\end{enumerate}

\section{Deep Learning At Scale}

\begin{minipage}[l]{.5\linewidth}
    \begin{figure}[H]
        \centering
        \subfigure{\fbox{\includegraphics[page=12, trim=0cm 0cm 0cm 0cm, clip, width=.95\linewidth]{L15_deeplearning_for_gpu_poor_part_1.pdf}}}
    \end{figure}    
\end{minipage}\hfill
\begin{minipage}[r]{.48\linewidth}
    \begin{itemize}
        \item For deeplearning we need model data and compute.
        \item At scale, you can't train the model many times for hyper-parmaeter tuning. A single model may take many dollars to train just once.
    \end{itemize}
\end{minipage}

\begin{minipage}[l]{.5\linewidth}
    \begin{figure}[H]
        \centering
        \subfigure{\fbox{\includegraphics[page=13, trim=0cm 0cm 0cm 0cm, clip, width=.95\linewidth]{L15_deeplearning_for_gpu_poor_part_1.pdf}}}
    \end{figure}    
\end{minipage}\hfill
\begin{minipage}[r]{.48\linewidth}
    ``When they have limitless compute and limitless datasets, they saw this really nice power law trend where it sticks very closely to the trend. This says that generally you can predict quite well, if you have limited compute and limited dataset size what loss you would get at the end.''
    The values here were proven to be wrong because the smaller models were not fully optimized; they used the same learning rate schedule as they did for the larger models. Therefore, the line was too steep.
\end{minipage}

\begin{minipage}[l]{.5\linewidth}
    \begin{figure}[H]
        \centering
        \subfigure{\fbox{\includegraphics[page=14, trim=0cm 0cm 0cm 0cm, clip, width=.95\linewidth]{L15_deeplearning_for_gpu_poor_part_1.pdf}}}
    \end{figure}    
\end{minipage}\hfill
\begin{minipage}[r]{.48\linewidth}
    ``openAI published finidngs about colour representing the size of the model, and they fit for given flops and compute budgets which model gets the lowest loss.'' 
    Here, we see that chinchilla is favoured since it prefers the number of data points over the model size.
\end{minipage}

\begin{minipage}[l]{.5\linewidth}
    \begin{figure}[H]
        \centering
        \subfigure{\fbox{\includegraphics[page=15, trim=0cm 0cm 0cm 0cm, clip, width=.95\linewidth]{L15_deeplearning_for_gpu_poor_part_1.pdf}}}
    \end{figure}    
\end{minipage}\hfill
\begin{minipage}[r]{.48\linewidth}
    Here we see that `deep learning has bene sovled'. ``This is showing that this happens in all modalities; these losses can just keep going until we get to super human performance.''

    \definition[Reducible Loss]{
        It is the actual loss - the irreductible loss, where irreducible loss is viewed as the true entropy of the data i.e. if I had a perfect fit of the data, what would be my loss even if I had a perfect fit?

        It can be viewed as the KL divergence between the mainfold of the dataspace and the manafold of the learnt model. 
    }
\end{minipage}

\begin{figure}[H]
    \centering
    \subfigure{\fbox{\includegraphics[page=16, trim=0cm 0cm 0cm 0cm, clip, width=.95\linewidth]{L15_deeplearning_for_gpu_poor_part_1.pdf}}}
    \caption{SORA AI just used more compute!}
\end{figure}    


\begin{minipage}[l]{.5\linewidth}
    \begin{figure}[H]
        \centering
        \subfigure{\fbox{\includegraphics[page=17, trim=0cm 0cm 0cm 0cm, clip, width=.95\linewidth]{L15_deeplearning_for_gpu_poor_part_1.pdf}}}
    \end{figure}    
\end{minipage}\hfill
\begin{minipage}[r]{.48\linewidth}
    ``When we plot all modalities the number of compute days vs the number of parameters for the optimal setting. There are correlation between what it is like to compute potimal size model for videa and for language model''
\end{minipage}

\begin{minipage}[l]{.5\linewidth}
    \begin{figure}[H]
        \centering
        \subfigure{\fbox{\includegraphics[page=18, trim=0cm 0cm 0cm 0cm, clip, width=.95\linewidth]{L15_deeplearning_for_gpu_poor_part_1.pdf}}}
    \end{figure}    
\end{minipage}\hfill
\begin{minipage}[r]{.48\linewidth}
    \begin{itemize}
        \item if blue is state of the art and red is my implementation,
        \item then we can do a scaling analysis and conclude that since the curve is steeper then it may perform better at high levels of compute. 
    \end{itemize}
\end{minipage}

\section{Methods for reducing computational requirements}

\subsection{Gradient Accumulation}

\begin{minipage}[l]{.5\linewidth}
    \begin{figure}[H]
        \centering
        \subfigure{\fbox{\includegraphics[page=19, trim=0cm 0cm 0cm 0cm, clip, width=.95\linewidth]{L15_deeplearning_for_gpu_poor_part_1.pdf}}}
    \end{figure}    
\end{minipage}\hfill
\begin{minipage}[r]{.48\linewidth}
    \begin{itemize}
        \item As we increase the batch size our computation requirements increase
        \item If we use 1 batch, this is just Stochastic Gradient Descent so you're approximating the true gradient with one sample; your loss will be jumping around wihtout convergence 
        \item \emph{accumulate the gradients without updating the optimizer, then when the mini-batch size times the number of cumulative gradient accumulation steps equals the target batch then we do the step in the loss surface.}
        \item Useful if you want a larger batch size
    \end{itemize}
\end{minipage}

\subsection{Gradient Checkpointing}

\begin{minipage}[l]{.5\linewidth}
    \begin{figure}[H]
        \centering
        \subfigure{\fbox{\includegraphics[page=20, trim=0cm 0cm 0cm 0cm, clip, width=.95\linewidth]{L15_deeplearning_for_gpu_poor_part_1.pdf}}}
    \end{figure}    
\end{minipage}\hfill
\begin{minipage}[r]{.48\linewidth}
    \begin{itemize}
        \item idea is to initially throw out the activation in the forward pass after we've done it, but in the backpropogation step we redo the entire forward computation step.
        \item This is computationally expensive, but the memory requirements stay low; it doesn't scale with the number of layers.
        \item The forward pass is much faster than the backwards pass
        \item Therefore, we select checkpoints from which we continue the forward pass. 
    \end{itemize}
\end{minipage}

\section{Finetuning problem setting}

\begin{minipage}[l]{.5\linewidth}
    \begin{figure}[H]
        \centering
        \subfigure[``For every downstream task'' we need to fine tune (lots of flops of updating) and store each separate model for each downstream task. This is computationally burdesome]{\fbox{\includegraphics[page=4, trim=0cm 0cm 0cm 0cm, clip, width=.95\linewidth]{L16_deeplearning_at_scale_part2.pdf}}}
    \end{figure}    
\end{minipage}\hfill
\begin{minipage}[r]{.48\linewidth}
    \begin{itemize}
        \item We're talking about autoregressive models; we have a pre-trained auto-regression model, and for our fine tuning datset, we select some context target pairs (the tokens before the token you're trying to predict).
        \item We're trying to min/max this over each of the pre-trained parameters. 
        \item This requires us to store teh gradients and optimisations for every single one of these parameteters.
        \item We also require more leaning signal to provide enough signal to update every single one of these parmaeters (scalability issue as shown above with scalling laws).
    \end{itemize}
\end{minipage}

\subsection{Low Rank Adaptation (LoRA)}

\begin{minipage}[l]{.5\linewidth}
    \begin{figure}[H]
        \centering
        \subfigure[Diagram: ``The intrinsic low rank of the over-parameterised space decreases as the number of parameters of our pre-trained model increases; the larger the model, the simpler the loss parameter space is''.]{\fbox{\includegraphics[page=5, trim=0cm 0cm 0cm 0cm, clip, width=.95\linewidth]{L16_deeplearning_at_scale_part2.pdf}}}
    \end{figure}    
\end{minipage}\hfill
\begin{minipage}[r]{.48\linewidth}
    \begin{itemize}
        \item During fine-tuning, researchers constrained the number of parameters that are available for fine-tuning and saw that the training performance actually didn't drop.
        \item If we increase the number of parameters, the likelihood of getting stuck in a local minima `saddle' increases.
        \item The authors: What happes if $\Delta \Phi$'s rank is intrinsicly low? We can instead update $\Theta$ that is significantly less than $\Phi$.
        \item TLDR: update only a subset of parameters.
    \end{itemize}
\end{minipage}

\begin{minipage}[l]{.5\linewidth}
    \begin{figure}[H]
        \centering
        \subfigure{\fbox{\includegraphics[page=6, trim=0cm 0cm 0cm 0cm, clip, width=.95\linewidth]{L16_deeplearning_at_scale_part2.pdf}}}
    \end{figure}    
\end{minipage}\hfill
\begin{minipage}[r]{.48\linewidth}
    \begin{itemize}
        \item If before, we learn that the pre-trained projection and also the finetuned projection $\Delta$ can we decompose $\Delta W$ into a low rank version. 
        \item in blue: the original pre-trained model where we take the input and multiply it by the weight matrix. We freeze these, so we don't learn anything about them.
        \item In green, in parallel we have an approximation of the weight matrix that is low rank. We only learn this.
        \item intuitively it works because we project $L_1 \times L_2$ into a $D \times D$ matrix.
    \end{itemize}
\end{minipage}

\begin{minipage}[l]{.5\linewidth}
    \begin{figure}[H]
        \centering
        \subfigure{\fbox{\includegraphics[page=7, trim=0cm 0cm 0cm 0cm, clip, width=.95\linewidth]{L16_deeplearning_at_scale_part2.pdf}}}
    \end{figure}    
\end{minipage}\hfill
\begin{minipage}[r]{.48\linewidth}
    \begin{itemize}
        \item they found that the low rank approximation of the weight matrix was able to capture the majority of the information in the original weight matrix even at $r=1$.
        \item This is susseptible to initialisation bias. They set $L_2$ to be 0s; they don't add any new information in the first couple of iterations -- its only the pre-trained model that is doing the work that is itsself forzen. Then the model slowly introduces the new information.
    \end{itemize}
\end{minipage}

\begin{minipage}[l]{.5\linewidth}
    \begin{figure}[H]
        \centering
        \subfigure{\fbox{\includegraphics[page=8, trim=0cm 0cm 0cm 0cm, clip, width=.95\linewidth]{L16_deeplearning_at_scale_part2.pdf}}}
    \end{figure}    
\end{minipage}\hfill
\begin{minipage}[r]{.48\linewidth}
    \begin{itemize}
        \item At inference time we have a lot of flops from multiplying the matrix together. What they do is they fuse the kernels togeher.
        \item We can then swap in the very small memory footprints in easily.
        \item We can then apply some methods from Gradient Accumulation/Checkpointing.
    \end{itemize}
\end{minipage}

Therefore the big drawback with LORA is that we still need to store the $W_0$ matrix.

\subsection{QLoRA}

\subsubsection{K-Bit Quantisation}

\begin{minipage}[l]{.5\linewidth}
    \begin{figure}[H]
        \centering
        \subfigure{\fbox{\includegraphics[page=9, trim=0cm 0cm 0cm 0cm, clip, width=.95\linewidth]{L16_deeplearning_at_scale_part2.pdf}}}
    \end{figure}    
\end{minipage}\hfill
\begin{minipage}[r]{.48\linewidth}
    \begin{itemize}
        \item K-bit quantisation: this is going from a data type of higher precision or a higher bit rate to a lower bit rate
        \item we accomplish this by scaling by the absolute max of the input data
    \end{itemize}
\end{minipage}

\begin{minipage}[l]{.5\linewidth}
    \begin{figure}[H]
        \centering
        \subfigure{\fbox{\includegraphics[page=10, trim=0cm 0cm 0cm 0cm, clip, width=.95\linewidth]{L16_deeplearning_at_scale_part2.pdf}}}
    \end{figure}    
\end{minipage}\hfill
\begin{minipage}[r]{.48\linewidth}
    \begin{itemize}
        \item We have an example with datapoints in float in blue. We then qunatise it down to int8
        \item We scale the absolute max to the lowest possible value and scale everything else equally by using the quantisation constant. 
        \item This is a lossy process. We can't go back to the original data, but we can go back to the original data type.
        \item This has the setback that if we have outliers, we scale things to a bin, and the represntation becomes very sparse.
        \item To solve this issue you perform chunks, into small pieces so that the chances of ther being an outlier is very small. 
        \item Each chunk therefore has a unique qunatisation constant. 
    \end{itemize}
\end{minipage}

\subsubsection{BrainFloat}

\begin{minipage}[l]{.5\linewidth}
    \begin{figure}[H]
        \centering
        \subfigure{\fbox{\includegraphics[page=11, trim=0cm 0cm 0cm 0cm, clip, width=.95\linewidth]{L16_deeplearning_at_scale_part2.pdf}}}
    \end{figure}    
\end{minipage}\hfill
\begin{minipage}[r]{.48\linewidth}
    \begin{itemize}
        \item This type favours the exponent significantly. 
        \item We found that the models were more robust to the actual precision of the bodies than they were to a thing called under or overflow. (where gradients are so small that they become 0)
        \item ``models performed significantly better when we worried more about underflow than overflow''.
    \end{itemize}
\end{minipage}

\subsection{Quantising $W_0$ from bfloat16 to 4 bit}

\subsubsection{4-bit NormalFloat}

\begin{minipage}[l]{.5\linewidth}
    \begin{figure}[H]
        \centering
        \subfigure{\fbox{\includegraphics[page=12, trim=0cm 0cm 0cm 0cm, clip, width=.95\linewidth]{L16_deeplearning_at_scale_part2.pdf}}}
    \end{figure}    
\end{minipage}\hfill
\begin{minipage}[r]{.48\linewidth}
    \begin{itemize}
        \item A big drawback with LORA is that we still need to store the $W_0$ matrix.
        \item We can use BrainFloat to reduce the memory footprint of the model to 4 bits!
        \item Here, we rely on the idea that most data in deep learning is a normal distribution. 
        \item Majority of values map to the same central point normally
        \item Therefore, they introduce a new type where values are normally distributed in the new type also. The idea is that they are wider at the edges
        \item The numbers are all the numbers that an Nfloat4 can represent, and in the graph we see Nfloat performs better.
    \end{itemize}
\end{minipage}

\subsubsection{Double Quantisation}

\begin{minipage}[l]{.5\linewidth}
    \begin{figure}[H]
        \centering
        \subfigure{\fbox{\includegraphics[page=13, trim=0cm 0cm 0cm 0cm, clip, width=.95\linewidth]{L16_deeplearning_at_scale_part2.pdf}}}
    \end{figure}    
\end{minipage}\hfill
\begin{minipage}[r]{.48\linewidth}
    \begin{itemize}
        \item In K-bit quantisation, we store quantisation constants that is stored in 32 bit precision. 
        \item If we quantise 65 billion parameters, then we have a lot of qunatisation constants.
        \item Therefore, we quantise the quantisation constant. 
        \item Quantisation and de-qunatisation is a cheap operation and fits nicely into consumer-grade hardware.
    \end{itemize}
\end{minipage}

\subsubsection{Dequnatisation}

\begin{minipage}[l]{.5\linewidth}
    \begin{figure}[H]
        \centering
        \subfigure{\fbox{\includegraphics[page=14, trim=0cm 0cm 0cm 0cm, clip, width=.95\linewidth]{L16_deeplearning_at_scale_part2.pdf}}}
    \end{figure}    
\end{minipage}\hfill
\begin{minipage}[r]{.48\linewidth}
    \begin{itemize}
        \item The noraml float is a static memory type. 
        \item When we train the model, the vast majority of the pre-trained model is stored in 4 bit.
        \item When we matrix multiply we de-quantise it into 16 bit brainfloat space and perform the matrix mulitplicaiton.
        \item ($L_1$ and $L_2$ are still in 16 bit representation)
    \end{itemize}
\end{minipage}

% \begin{minipage}[l]{.5\linewidth}
%     \begin{figure}[H]
%         \centering
%         \subfigure{\fbox{\includegraphics[page=15, trim=0cm 0cm 0cm 0cm, clip, width=.95\linewidth]{L16_deeplearning_at_scale_part2.pdf}}}
%     \end{figure}    
% \end{minipage}\hfill
% \begin{minipage}[r]{.48\linewidth}
%     \begin{itemize}
%         \item
%     \end{itemize}
% \end{minipage}

% \begin{minipage}[l]{.5\linewidth}
%     \begin{figure}[H]
%         \centering
%         \subfigure{\fbox{\includegraphics[page=16, trim=0cm 0cm 0cm 0cm, clip, width=.95\linewidth]{L16_deeplearning_at_scale_part2.pdf}}}
%     \end{figure}    
% \end{minipage}\hfill
% \begin{minipage}[r]{.48\linewidth}
%     \begin{itemize}
%         \item
%     \end{itemize}
% \end{minipage}

% \begin{minipage}[l]{.5\linewidth}
%     \begin{figure}[H]
%         \centering
%         \subfigure{\fbox{\includegraphics[page=17, trim=0cm 0cm 0cm 0cm, clip, width=.95\linewidth]{L16_deeplearning_at_scale_part2.pdf}}}
%     \end{figure}    
% \end{minipage}\hfill
% \begin{minipage}[r]{.48\linewidth}
%     \begin{itemize}
%         \item
%     \end{itemize}
% \end{minipage}

% \begin{minipage}[l]{.5\linewidth}
%     \begin{figure}[H]
%         \centering
%         \subfigure{\fbox{\includegraphics[page=18, trim=0cm 0cm 0cm 0cm, clip, width=.95\linewidth]{L16_deeplearning_at_scale_part2.pdf}}}
%     \end{figure}    
% \end{minipage}\hfill
% \begin{minipage}[r]{.48\linewidth}
%     \begin{itemize}
%         \item
%     \end{itemize}
% \end{minipage}

% \begin{minipage}[l]{.5\linewidth}
%     \begin{figure}[H]
%         \centering
%         \subfigure{\fbox{\includegraphics[page=19, trim=0cm 0cm 0cm 0cm, clip, width=.95\linewidth]{L16_deeplearning_at_scale_part2.pdf}}}
%     \end{figure}    
% \end{minipage}\hfill
% \begin{minipage}[r]{.48\linewidth}
%     \begin{itemize}
%         \item
%     \end{itemize}
% \end{minipage}

% \begin{minipage}[l]{.5\linewidth}
%     \begin{figure}[H]
%         \centering
%         \subfigure{\fbox{\includegraphics[page=20, trim=0cm 0cm 0cm 0cm, clip, width=.95\linewidth]{L16_deeplearning_at_scale_part2.pdf}}}
%     \end{figure}    
% \end{minipage}\hfill
% \begin{minipage}[r]{.48\linewidth}
%     \begin{itemize}
%         \item
%     \end{itemize}
% \end{minipage}

% \begin{minipage}[l]{.5\linewidth}
%     \begin{figure}[H]
%         \centering
%         \subfigure{\fbox{\includegraphics[page=21, trim=0cm 0cm 0cm 0cm, clip, width=.95\linewidth]{L16_deeplearning_at_scale_part2.pdf}}}
%     \end{figure}    
% \end{minipage}\hfill
% \begin{minipage}[r]{.48\linewidth}
%     \begin{itemize}
%         \item
%     \end{itemize}
% \end{minipage}

% \begin{minipage}[l]{.5\linewidth}
%     \begin{figure}[H]
%         \centering
%         \subfigure{\fbox{\includegraphics[page=22, trim=0cm 0cm 0cm 0cm, clip, width=.95\linewidth]{L16_deeplearning_at_scale_part2.pdf}}}
%     \end{figure}    
% \end{minipage}\hfill
% \begin{minipage}[r]{.48\linewidth}
%     \begin{itemize}
%         \item
%     \end{itemize}
% \end{minipage}

% \begin{minipage}[l]{.5\linewidth}
%     \begin{figure}[H]
%         \centering
%         \subfigure{\fbox{\includegraphics[page=23, trim=0cm 0cm 0cm 0cm, clip, width=.95\linewidth]{L16_deeplearning_at_scale_part2.pdf}}}
%     \end{figure}    
% \end{minipage}\hfill
% \begin{minipage}[r]{.48\linewidth}
%     \begin{itemize}
%         \item
%     \end{itemize}
% \end{minipage}

% \begin{minipage}[l]{.5\linewidth}
%     \begin{figure}[H]
%         \centering
%         \subfigure{\fbox{\includegraphics[page=24, trim=0cm 0cm 0cm 0cm, clip, width=.95\linewidth]{L16_deeplearning_at_scale_part2.pdf}}}
%     \end{figure}    
% \end{minipage}\hfill
% \begin{minipage}[r]{.48\linewidth}
%     \begin{itemize}
%         \item
%     \end{itemize}
% \end{minipage}

% \begin{minipage}[l]{.5\linewidth}
%     \begin{figure}[H]
%         \centering
%         \subfigure{\fbox{\includegraphics[page=25, trim=0cm 0cm 0cm 0cm, clip, width=.95\linewidth]{L16_deeplearning_at_scale_part2.pdf}}}
%     \end{figure}    
% \end{minipage}\hfill
% \begin{minipage}[r]{.48\linewidth}
%     \begin{itemize}
%         \item
%     \end{itemize}
% \end{minipage}

% \begin{minipage}[l]{.5\linewidth}
%     \begin{figure}[H]
%         \centering
%         \subfigure{\fbox{\includegraphics[page=26, trim=0cm 0cm 0cm 0cm, clip, width=.95\linewidth]{L16_deeplearning_at_scale_part2.pdf}}}
%     \end{figure}    
% \end{minipage}\hfill
% \begin{minipage}[r]{.48\linewidth}
%     \begin{itemize}
%         \item
%     \end{itemize}
% \end{minipage}

% \begin{minipage}[l]{.5\linewidth}
%     \begin{figure}[H]
%         \centering
%         \subfigure{\fbox{\includegraphics[page=27, trim=0cm 0cm 0cm 0cm, clip, width=.95\linewidth]{L16_deeplearning_at_scale_part2.pdf}}}
%     \end{figure}    
% \end{minipage}\hfill
% \begin{minipage}[r]{.48\linewidth}
%     \begin{itemize}
%         \item
%     \end{itemize}
% \end{minipage}

% \begin{minipage}[l]{.5\linewidth}
%     \begin{figure}[H]
%         \centering
%         \subfigure{\fbox{\includegraphics[page=28, trim=0cm 0cm 0cm 0cm, clip, width=.95\linewidth]{L16_deeplearning_at_scale_part2.pdf}}}
%     \end{figure}    
% \end{minipage}\hfill
% \begin{minipage}[r]{.48\linewidth}
%     \begin{itemize}
%         \item
%     \end{itemize}
% \end{minipage}

% \begin{minipage}[l]{.5\linewidth}
%     \begin{figure}[H]
%         \centering
%         \subfigure{\fbox{\includegraphics[page=29, trim=0cm 0cm 0cm 0cm, clip, width=.95\linewidth]{L16_deeplearning_at_scale_part2.pdf}}}
%     \end{figure}    
% \end{minipage}\hfill
% \begin{minipage}[r]{.48\linewidth}
%     \begin{itemize}
%         \item
%     \end{itemize}
% \end{minipage}

% \begin{minipage}[l]{.5\linewidth}
%     \begin{figure}[H]
%         \centering
%         \subfigure{\fbox{\includegraphics[page=30, trim=0cm 0cm 0cm 0cm, clip, width=.95\linewidth]{L16_deeplearning_at_scale_part2.pdf}}}
%     \end{figure}    
% \end{minipage}\hfill
% \begin{minipage}[r]{.48\linewidth}
%     \begin{itemize}
%         \item
%     \end{itemize}
% \end{minipage}

% \printbibliography
% \addcontentsline{toc}{section}{Bibliography}

\end{document}